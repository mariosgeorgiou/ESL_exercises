\documentclass[]{book}

%These tell TeX which packages to use.
\usepackage{array,epsfig}
\usepackage{amsmath}
\usepackage{amsfonts}
\usepackage{amssymb}
\usepackage{amsxtra}
\usepackage{amsthm}
\usepackage{mathrsfs}
\usepackage{color}
\usepackage{cleveref}

%Here I define some theorem styles and shortcut commands for symbols I use often
\theoremstyle{definition}
\newtheorem{defn}{Definition}
\newtheorem{thm}{Theorem}
\newtheorem{cor}{Corollary}
\newtheorem*{rmk}{Remark}
\newtheorem{lem}{Lemma}
\newtheorem*{joke}{Joke}
\newtheorem{ex}{Example}
\newtheorem*{soln}{Solution}
\newtheorem{prop}{Proposition}

\newcommand{\lra}{\longrightarrow}
\newcommand{\ra}{\rightarrow}
\newcommand{\surj}{\twoheadrightarrow}
\newcommand{\graph}{\mathrm{graph}} \newcommand{\bb}[1]{\mathbb{#1}}
\newcommand{\Z}{\bb{Z}} \newcommand{\Q}{\bb{Q}} \newcommand{\R}{\bb{R}}
\newcommand{\C}{\bb{C}} \newcommand{\N}{\bb{N}} \newcommand{\M}{\mathbf{M}}
\newcommand{\m}{\mathbf{m}} \newcommand{\MM}{\mathscr{M}}
\newcommand{\HH}{\mathscr{H}}
\newcommand{\Om}{\Omega}
\newcommand{\Ho}{\in\HH(\Om)}
\newcommand{\bd}{\partial}
\newcommand{\del}{\partial}
\newcommand{\bardel}{\overline\partial}
\newcommand{\textdf}[1]{\textbf{\textsf{#1}}\index{#1}}
\newcommand{\img}{\mathrm{img}}
\newcommand{\ip}[2]{\left\langle{#1},{#2}\right\rangle}
\newcommand{\inter}[1]{\mathrm{int}{#1}}
\newcommand{\exter}[1]{\mathrm{ext}{#1}} \newcommand{\cl}[1]{\mathrm{cl}{#1}}
\newcommand{\ds}{\displaystyle}
\newcommand{\vol}{\mathrm{vol}} \newcommand{\cnt}{\mathrm{ct}}
\newcommand{\osc}{\mathrm{osc}} \newcommand{\LL}{\mathbf{L}}
\newcommand{\UU}{\mathbf{U}} \newcommand{\support}{\mathrm{support}}
\newcommand{\AND}{\;\wedge\;}
\newcommand{\OR}{\;\vee\;}
\newcommand{\Oset}{\varnothing}
\newcommand{\st}{\ni}
\newcommand{\wh}{\widehat}

\DeclareMathOperator*{\argmax}{arg\,max}
\DeclareMathOperator*{\argmin}{arg\,min}


%Pagination stuff.
\setlength{\topmargin}{-.3 in}
\setlength{\oddsidemargin}{0in}
\setlength{\evensidemargin}{0in}
\setlength{\textheight}{9.in}
\setlength{\textwidth}{6.5in}
\pagestyle{empty}



\begin{document}


\begin{center}
	{\Large Math 3220-1 \hspace{0.5cm} HW 1}\\
	\textbf{NAME}\\ %You should put your name here
	Due: DATE %You should write the date here.
\end{center}

\vspace{0.2 cm}


\subsection*{Exercises for Section 2}

\begin{enumerate}
	\item\label{k-classes} Suppose each of $K$-classes has an associated target
	$t_k$, which is a vector of all zeros, except a one in the $k$th position.
	Show that classifying to the largest element of $\hat{y}$ amounts to
	choosing the closest target, $\min_k\|t_k-\hat{y}\|$, if the elements of
	$\hat{y}$ sum to one.
	\begin{soln}
		\newcommand{\normone}[1]{\sum_{i\ne #1}|\hat{y}_i|+|1-\hat{y}_{#1}|} Let
		$k^*=\argmax_k \hat{y}_k$ and suppose that there is $k'\le k^*$ such
		that $\|t_{k'}-\hat{y}\| < \|t_{k^*}-\hat{y}\|$.
		\begin{itemize}
			\item $\ell_1$ norm. It holds that
			      $\|t_k-\hat{y}\|_1=\sum_i|t_{k,i}-\hat{y}_i|=\sum_{i\ne
			      k}|\hat{y}_i|+|1-\hat{y}_k|$. Hence, we get
			      \begin{equation}\label{2.1-inequality}
				      \normone{k'} < \normone{k^*}\Rightarrow |\hat{y}_{k^*}|-|1-\hat{y}_{k^*}|
				      < |\hat{y}_{k'}|-|1-\hat{y}_{k'}|.
			      \end{equation}
			      But the function $f(y)=|y|-|1-y|$ is increasing in $[0,1]$
			      hence~\Cref{2.1-inequality} implies that
			      $\hat{y}_{k^*}<\hat{y}_{k'}$, reaching a contradiction.
			\item $\ell_2$ norm. Similarly, we get that
			      $\hat{y}_{k^*}(1-\hat{y}_{k^*})<\hat{y}_{k'}(1-\hat{y}_{k'})$
			      and since the function $f(y)=y(1-y)$ is increasing in $[0,1]$,
			      we get that $\hat{y}_{k^*}<\hat{y}_{k'}$, reaching a
			      contradiction.
		\end{itemize}
	\end{soln}

\end{enumerate}



\end{document}


